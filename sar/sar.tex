\documentclass{article}

\usepackage{amsmath,amssymb,amsthm}
\usepackage{url}
\usepackage{hyperref}
\usepackage{proof}
\usepackage{stmaryrd}
\usepackage{xspace}
%\usepackage{MnSymbol}
\usepackage{parskip}
\usepackage{fullpage}
\usepackage{mathpartir}
\usepackage{subcaption}
\usepackage{float}


\begin{document}

\newcommand{\interp}[1]{\llbracket #1 \rrbracket}
\newcommand{\rinterp}[1]{\llparenthesis #1 \rrparenthesis}
\newcommand{\lft}[1]{\llceil #1 \rrceil}
\newcommand{\gprj}[1]{| #1 |}
\newtheorem{theorem}{Theorem}
\newtheorem{lemma}[theorem]{Lemma}
\newtheorem{corollary}[theorem]{Corollary}
\newtheorem{definition}[theorem]{Definition}
\newtheorem{notation}[theorem]{Notation}
\newtheorem{proposition}[theorem]{Proposition}
%\theoremstyle{definition}
\newtheorem{example}{Example}[section]
\newtheorem{remark}{Remark}[section]
\newcommand{\lftcc}[0]{$\lft{\texttt{CC}}$\xspace}
\newcommand{\infcc}[0]{$\texttt{CC}_\infty$\xspace}
\newcommand{\Pia}[3]{\Pi\, #1: #2\, .\, #3}
\newcommand{\Pik}[4]{\Pi\, #1\langle #2 \rangle : #3\, .\, #4}
\newcommand{\all}[2]{\forall\, #1\,.\, #2}
\newcommand{\alll}[3]{\forall^{#1}\, #2\,.\, #3}
\newcommand{\lam}[2]{\lambda\, #1\, .\, #2}
\newcommand{\tlam}[3]{\lambda\, #1 : #2\, .\, #3}
\newcommand{\Tlam}[3]{\Lambda\, #1 : #2\, .\, #3}
\newcommand{\capl}[0]{\stackrel{\cdot}{\cap}}
\newcommand{\ltl}[0]{\lambda 2^{\mathcal{L}}}
\newcommand{\tth}[3]{\theta\, #1 : #2\, .\, #3}
\newcommand{\univ}[0]{\mathcal{U}}
\newcommand{\subst}[3]{[#1 / #2] #3}
\newcommand{\betaeq}[2]{#1 \mathop{=_{\beta}} #2}
\newcommand{\reftc}[1]{\RefTirName{#1}}
\newcommand{\wrec}[2]{#1\ \bullet\ #2}
\newcommand{\ctor}[3]{[ #1 , #2, #3 ]}
\newcommand{\wunit}[0]{\langle\rangle}

\newcommand{\sar}[0]{SAR\xspace}
\newcommand{\eqtm}[3]{#2\{#1 \mathrel{\operatorname{=}} #3\}}
\newcommand{\rfl}[1]{\epsilon\ #1}
\newcommand{\cng}[4]{\rho\, [#1\,.\,#2]\ #3 - #4}
\newcommand{\ext}[1]{\xi\, #1}

\newcommand{\rrec}[0]{\mathsf{R}_{\textit{Nat}}}
\newcommand{\w}[3]{\langle #1 , #2 , #3 \rangle}
\newcommand{\wb}[3]{\ll #1 , #2 , #3 \gg} 
\newcommand{\wc}[3]{\langle| #1 , #2 , #3 |\rangle}

\newcommand{\inj}[2]{\iota_#1\ #2}

\newcommand{\selfa}[1]{\sigma\,#1}
\newcommand{\absta}[1]{\tau\,#1}

\title{Type in Type and Schematic Affine Recursion}

\author{Aaron Stump and Victor Taelin}

\maketitle

\section{Terminating Computation First}

Constructive type theories based on the Curry-Howard isomorphism
enforce logical soundness by ensuring that all programs are uniformly
terminating.  Proofs are identified with programs, and diverging
programs would constitute proofs of any proposition.  So these must be
ruled out statically.  The approach adopted in systems like Coq, Agda,
and Lean is to enforce termination through a combination of typing
and syntactic checks for structural decrease at recursive calls.

This paper proposes an alternative, where termination is enforced
through syntactic checks alone, prior to typing.  The approach has the
drawback that programs must be significantly restricted in order to
guarantee termination without any reference to types.  There is a
notable benefit, however: the type system is now no longer required to
enforce termination.  This greatly increases the options for typing,
which now needs only to satisfy type safety, as required in
Programming Languages.

We present a language \sar, which combines an untyped affine lambda
calculus with a form of structural recursion.  With no further
restriction, this language allows diverging terms.  So we impose what
Alves et al. call the ``closed-by-construction'' restriction on
structural recursion, found also in~\cite{dallago09}.  This requires
that the functions to be iterated when recursing are closed.  But this
rules out most of the usual higher-order functions like \texttt{map}
on lists, where the function to iterate calls a function $f$ that
is given as a variable bound outside the recursion.  We address this
problem by proposing a language of schematic terms, where such variables
$f$ are not $\lambda$-bound, but treated schematically.  This allows
generic definition of functions like \texttt{map}, without losing
termination.

With termination of \sar established prior to typing, we are free to
adopt a more exotic type system than possible in other constructive
type theories, where the burden of termination falls on typing.  To
demonstrate this, we consider a dependent type system called Typed
\sar, with the ``\textit{Type} : \textit{Type}'' principle.

\section{Previous work on termination with affine recursion}

\section{Schematic affine recursion}

\section{Typing without termination}

\bibliographystyle{plain}
\bibliography{main}

\end{document}

\documentclass{article}

\usepackage{amsmath,amssymb,amsthm}
\usepackage{url}
\usepackage{hyperref}
\usepackage{proof}
\usepackage{stmaryrd}
\usepackage{xspace}
%\usepackage{MnSymbol}
\usepackage{parskip}
\usepackage{fullpage}
\usepackage{mathpartir}
\usepackage{subcaption}
\usepackage{float}


\begin{document}

\newcommand{\interp}[1]{\llbracket #1 \rrbracket}
\newcommand{\rinterp}[1]{\llparenthesis #1 \rrparenthesis}
\newcommand{\lft}[1]{\llceil #1 \rrceil}
\newcommand{\gprj}[1]{| #1 |}
\newtheorem{theorem}{Theorem}
\newtheorem{lemma}[theorem]{Lemma}
\newtheorem{corollary}[theorem]{Corollary}
\newtheorem{definition}[theorem]{Definition}
\newtheorem{notation}[theorem]{Notation}
\newtheorem{proposition}[theorem]{Proposition}
%\theoremstyle{definition}
\newtheorem{example}{Example}[section]
\newtheorem{remark}{Remark}[section]
\newcommand{\lftcc}[0]{$\lft{\texttt{CC}}$\xspace}
\newcommand{\infcc}[0]{$\texttt{CC}_\infty$\xspace}
\newcommand{\Pia}[3]{\Pi\, #1: #2\, .\, #3}
\newcommand{\Pik}[4]{\Pi\, #1\langle #2 \rangle : #3\, .\, #4}
\newcommand{\all}[2]{\forall\, #1\,.\, #2}
\newcommand{\alll}[3]{\forall^{#1}\, #2\,.\, #3}
\newcommand{\lam}[2]{\lambda\, #1\, .\, #2}
\newcommand{\tlam}[3]{\lambda\, #1 : #2\, .\, #3}
\newcommand{\Tlam}[3]{\Lambda\, #1 : #2\, .\, #3}
\newcommand{\capl}[0]{\stackrel{\cdot}{\cap}}
\newcommand{\ltl}[0]{\lambda 2^{\mathcal{L}}}
\newcommand{\tth}[3]{\theta\, #1 : #2\, .\, #3}
\newcommand{\univ}[0]{\mathcal{U}}
\newcommand{\subst}[3]{[#1 / #2] #3}
\newcommand{\betaeq}[2]{#1 \mathop{=_{\beta}} #2}
\newcommand{\reftc}[1]{\RefTirName{#1}}
\newcommand{\wrec}[2]{#1\ \bullet\ #2}
\newcommand{\ctor}[3]{[ #1 , #2, #3 ]}
\newcommand{\wunit}[0]{\langle\rangle}

\newcommand{\sar}[0]{SAR\xspace}
\newcommand{\eqtm}[3]{#2\{#1 \mathrel{\operatorname{=}} #3\}}
\newcommand{\rfl}[1]{\epsilon\ #1}
\newcommand{\cng}[4]{\rho\, [#1\,.\,#2]\ #3 - #4}
\newcommand{\ext}[1]{\xi\, #1}

\newcommand{\rrec}[0]{\mathsf{R}_{\textit{Nat}}}
\newcommand{\w}[3]{\langle #1 , #2 , #3 \rangle}
\newcommand{\wb}[3]{\ll #1 , #2 , #3 \gg} 
\newcommand{\wc}[3]{\langle| #1 , #2 , #3 |\rangle}

\newcommand{\inj}[2]{\iota_#1\ #2}

\title{Type in Type and Schematic Affine Recursion}

\author{Aaron Stump and Victor Taelin}

\maketitle

\section{Terminating Computation First}

Constructive type theories based on the Curry-Howard isomorphism
enforce logical soundness by ensuring that all programs are uniformly
terminating.  Proofs are identified with programs, and diverging
programs would thus prove arbitrary propositions.  So these must be
ruled out statically.  The approach adopted in systems like Coq, Agda,
and Lean is to enforce termination through a combination of typing
and syntactic checks for structural decrease at recursive calls.

This paper proposes an alternative, where termination is enforced
through syntactic checks alone, prior to typing.  The approach has the
drawback that programs must be significantly restricted in order to
guarantee termination without any reference to types.  There is a
notable benefit, however: the type system is now no longer required to
enforce termination.  This greatly increases the options for the
typing relation, which now needs only to satisfy type safety, as
required in Programming Languages.

We present a language \sar, which combines an untyped affine lambda
calculus with a form of structural recursion.  With no further
restriction, this language allows diverging terms.  So we impose what
Alves et al. call the ``closed-by-construction'' restriction on
structural recursion, found also in~\cite{dallago09}.  This requires
that the functions to be iterated when recursing are closed.  But this
rules out most of the usual higher-order functions like \texttt{map}
on lists, where the function to iterate calls a function $f$ that
is given as a variable bound outside the recursion.  We address this
problem by proposing a language of schematic terms, where such variables
$f$ are not $\lambda$-bound, but treated schematically.  This allows
generic definition of functions like \texttt{map}, without losing
termination.

With termination of \sar established prior to typing, we are free to
adopt a more exotic type system than possible in other constructive
type theories, where the burden of termination falls on typing.  To
demonstrate this, we consider a dependent type system called Typed
\sar, with the ``\textit{Type} : \textit{Type}'' principle.

\section{Previous work on termination with affine recursion}

Affine lambda calculus restricts $\lambda$-abstractions so that the
$\lambda$-bound variable may occur at most once in the body of the
abstraction.  With this restriction, $\beta$-reduction is easily seen
to be terminating, as the number of applications decreases by at least
one with every $\beta$-step.

\begin{figure}
  \[
  \begin{array}{llll}
    \textit{Terms} & t & ::= & x\ |\ \lam{x}{t}\ |\ t\ t'\ |\ \mathsf{Nil}\ |\ \mathsf{Cons}\ t_1\ t_2\ |\ \mathsf{R}\ t_1\ t_2\ t_3
  \end{array}
  \]
  \caption{The syntax of untyped terms}
\label{fig:syn}
\end{figure}

But affine lambda calculus seems too restrictive for regular
programming.  For example, Church-encoded data such as natural numbers
and lists are not affine in general.  So we consider expanding the
language with some form of inductive datatype, and its structural
recursor.  For simplicity, in this paper we just consider a datatype of lists, with its
recursor.  The syntax is shown in Figure~\ref{fig:syn}, and its reduction semantics in Figure~\ref{fig:redsem}.

\begin{figure}
  \[
  \begin{array}{llll}
    (\lam{x}{t})\ t' & \leadsto & [t'/x]t \\
    \mathsf{R}\ \mathsf{Nil}\ t_1\ t_2 & \leadsto & t_2 \\
    \mathsf{R}\ (\mathsf{Cons}\ t_a\ t_b)\ t_1\ t_2 & \leadsto & t_1\ t_a\ t_b\ (\mathsf{R}\ t_b\ t_1\ t_2)
  \end{array}
  \]
  \caption{Reduction semantics}
\label{fig:redsem}
\end{figure}

\noindent Without some further restriction, this language is
easily seen to allow diverging terms.  Let us see an example, adapted
from~\cite{alves10}.  Define
\begin{eqnarray*}
  \textit{app}_t & = & \lam{y}{y\ t} \\
  \textit{apply} & = & \lam{a}{\lam{b}{a\ b}} \\
\end{eqnarray*}
\noindent where $\textit{app}_t$ is schematic in meta-variable $t$. Then we have this reduction sequence, for any term $t$:
\[
\begin{array}{ll}
  \textit{app}_t\ (\textit{app}_t\ \textit{apply}) & \leadsto^2 \\
  \textit{app}_t\ (\lam{b}{t\ b}) & \leadsto^2 \\
  t\ t& \
  \end{array}
\]
\noindent The example is easier to complete with an iterator, defined as
\[
\textit{It} = \lam{n}{\lam{f}{\lam{x}{\mathsf{R}\ n\ (\lam{q}{\lam{r}{f}})\ x}}}
\]
\noindent This uses \textsf{R} to repeat the function $f$, but dropping the head
and tail of the list that \textsf{R} supplies to its second argument in case the first is a \textsf{Cons}.  Using \textit{It}, define
\begin{eqnarray*}
\textit{T} & = & \mathsf{Cons}\ u\ (\mathsf{Cons}\ u\ \mathsf{Nil}) \\
\Delta & = & \lam{x}{\textit{It}\ T\ \textit{app}_x\ \textit{apply}}
\end{eqnarray*}

\noindent \textit{T} is a list of length two, and applying \textit{It} with it
will lead to two nested calls.  So we have
\[
\begin{array}{ll}
  \textit{It}\ s\ \textit{app}_x\ \textit{apply} & \leadsto^+ \\
  \textit{app}_x\ (\textit{app}_x\ \textit{apply}) & \leadsto^+ \\
  x\ x& \
\end{array}
\]
\noindent So the term $\Omega = \Delta\ \Delta$ is not normalizing, 
because $\Delta$ reduces to $\lam{x}{x\ x}$.

This example is not so surprising, since the reduction rule for
\textsf{R} is (at the meta-level) not affine.  What is more surprising
is that the example still is diverging if one adopts what Alves et al.
call the ``closed at reduction'' restriction on the reduction
rules for \textsf{R}:
  \[
  \begin{array}{lllll}
    \mathsf{R}\ \mathsf{Nil}\ t_1\ t_2 & \leadsto & t_2, & \textnormal{if } \textit{FV}(t_1) = \emptyset\\
    \mathsf{R}\ (\mathsf{Cons}\ t_a\ t_b)\ t_1\ t_2 & \leadsto & t_1\ t_a\ t_b\ (\mathsf{R}\ t_b\ t_1\ t_2),& \textnormal{if } \textit{FV}(t_1) = \emptyset
  \end{array}
  \]
\noindent With the ``closed at reduction'' restriction, the former
reduction sequence is not available.  But we still have this one:
\[
\begin{array}{ll}
  \Delta\ \Delta & \leadsto \\
  \textit{It}\ T\ \textit{app}_\Delta\ \textit{apply} & \leadsto^+ \\
  \textit{app}_\Delta\ (\textit{app}_\Delta\ \textit{apply}) & \leadsto^+ \\
  \Delta\ \Delta& \
\end{array}
\]
\noindent showing that $\Omega$ diverges.

Alves et al. observe that if one goes even further and restricts the
syntax so that $\mathsf{R}\ t\ t_1\ t_2$ is only syntactically allowed
if $t_1$ is closed, then the language is indeed terminating.  They
call this the ``closed at construction'' restriction.  It
significantly reduces the computational power of the language: only
the primitive recursive functions are definable~\cite{alves10}.

While limiting oneself to primitive recursive functions may be a
concern for theoretical applications, it is not a concern for
practical programming, where the primitive recursive functions already
encompass computations far beyond the feasible.  But ``closed at construction''
does have a serious practical drawback: it prevents the usual higher-order
combinators one expects in functional programming.  For example, using the
unrestricted syntax, we may define a \texttt{map} function on lists:
\[
\texttt{map} = \lam{f}{\lam{x}{\mathsf{R}\ x\ (\lam{h}{\lam{t}{\lam{r}{\mathsf{Cons}\ (f\ h)\ r}}})\ \mathsf{Nil}}}
\]
\noindent Here, \textsf{R} is used to recurse through list $x$,
applying $f$ to the head $h$ of each sublist.  But the term performing
that application has $f$ free.  So this program, along with many
others from usual functional programming practice, will be disallowed
by affine lambda calculus with ``closed at construction'' recursion.

\section{Schematic affine recursion}

We can address this difficulty by applying a basic idea that one finds
in both Logic and Programming Languages, namely the use of schematic
constructions.  For the definition of Peano Arithmetic in first-order
logic, one postulates a \emph{scheme} of induction.  This is a
meta-level formulation representing an infinite set of axioms, each
one expressing the induction principle for proving $\all{x}{\phi}$
from base and step cases.  This is in contrast to second-order logic,
where a single axiom suffices, by quantifying, within the logic, over
the formula $\phi$.  In contrast, the first-order scheme of induction
is a meta-level quantification, outside the logic, over $\phi$.

We can use this same method here to recover generic programming while
respecting the ``closed at construction'' requirement.  We extend the
syntax of Figure~\ref{fig:syn} to allow top-level schematic
definitions of terms.  This is shown in Figure~\ref{fig:sarsyn}.
Schematic terms have the same syntax as terms above, except
for the addition of the construct $\mathbf{f}[\bar{t}]$.  Here,
$\bar{t}$ is a (finite) vector of schematic terms, and $\mathbf{f}$ is
a defined symbol.  Figure~\ref{fig:sarwf} defines a relation $\textit{Wf}$
on lists of definitions, imposing these requirements:
\begin{itemize}
\item The defined symbols occurring on the right-hand side of a definition
  are defined earlier in the list, with
  the same number of schematic variables $\bar{u}$ as the arguments $\bar{t}$
  where that symbol is applied.
\item In applications $\mathbf{f}[\bar{t}]$, the terms $\bar{t}$ are
  all required to be closed, which means that they have no free variables
  except for schematic ones.  Those will only be instantiated by closed
  terms, so they may stand for such in applications of defined symbols.
\item The term $t_1$ in a recursion $\mathsf{R}\ t_1\ t_2\ t_3$ is
  required to be closed.
  \end{itemize}
These requirements are imposed using the following relations:
\begin{itemize}
\item $\textit{Wf}\ \Delta$ expresses that the list $\Delta$ of
  schematic definitions is well-formed, in the sense that
each definition in $\Delta$ is well-formed with respect to the definitions
to the left of it in $\Delta$.
\item The relation $\Delta \vdash \textit{Wf}\ D$ expresses
  that definition $D$ is well-formed with respect to $\Delta$.  This requires, for definition
  $\mathbf{f}[\bar{u}] = t$, that the body $t$ of the
  definition is closed with respect to the parameters $\bar{u}$, and
  (for simplicity) that $\mathbf{f}$ is not declared already in $\Delta$.
\item The relation $\Delta; \bar{u}\ |\ \bar{x} \vdash
  \textit{Closed}\ t$ means that term $t$ is closed with respect to
  the definitions in $\Delta$ (which are just used to check that
  defined symbols $\mathbf{f}$ are applied to the correct number of
  arguments), and the parameters $\bar{u}$.  The parameters are
  considered closed themselves, as are the locally bound variables
  $\bar{x}$, except that those variables are dropped in premises
  (of the rules of Figure~\ref{fig:wf}) where
  terms must not contain any free local variables.  This can be seen
  where $\bar{x}$ is replaced with $\cdot$ in premises to the rules
  expressing closedness of $\mathsf{R}$-terms and applications
  $\mathbf{f}[\bar{t}]$ of defined symbols.
\end{itemize}

\begin{figure}
  \[
  \begin{array}{llll}
   \textit{Defined symbols} & \mathbf{f} & \ &\ \\
   \textit{Schematic variables} & u & \ &\ \\
    \textit{Definitions} & D & ::= & \mathbf{f}[\bar{u}] = t \\
    \textit{Schematic terms} & t & ::= & x\ |\ u\ |\ \mathbf{f}[\bar{t}]\ |\ \lam{x}{t}\ |\ t\ t'\ |\\
    \ &\ &\ & \mathsf{Nil}\ |\ \mathsf{Cons}\ t_1\ t_2\ |\ \mathsf{R}\ t_1\ t_2\ t_3 \\
    \textit{Lists of definitions} & \Delta & ::= & \cdot \ |\ \Delta, D
  \end{array}
  \]
  \caption{Syntax of \sar}
\label{fig:sarsyn}
\end{figure}

\begin{figure}
  \[
  \begin{array}{lll}
    \infer{\textit{Wf}\ \cdot}{\ } &
    \infer{\textit{Wf}\ (\Delta,\ D)}{\textit{Wf}\ \Delta & \Delta \vdash \textit{Wf}\ D}
    &\
    \\ \\
    \infer{\Delta\vdash\textit{Wf}\ \ \mathbf{f}[\bar{u}] = t}
          {\begin{array}{l}
              \Delta; \bar{u}\ |\ \cdot \vdash\textit{Closed}\ t\\
          \all{\bar{u}'}{\all{t'}{(\mathbf{f}[\bar{u}'] = t')\not\in \Delta}} \end{array}} &
    \infer{\Delta; \bar{u}\ |\ \bar{x} \vdash\textit{Closed}\ u }{u \in \{\bar{u}\}} &
    \infer{\Delta; \bar{u}\ |\ \bar{x} \vdash\textit{Closed}\ x }{x \in \{\bar{x}\}} 
  \\ \\           
  \infer{\Delta;\bar{u}\ |\ \bar{x} \vdash\textit{Closed}\ \lam{y}{t}}{\Delta;\bar{u}\ |\ \bar{x},y \vdash\textit{Closed}\ t} &
  \infer{\Delta;\bar{u}\ |\ \bar{x} \vdash\textit{Closed}\ (t_1\ t_2)}{\begin{array}{l}\Delta;\bar{u}\ |\ \bar{x} \vdash\textit{Closed}\ t_1\\
                                                                                       \Delta;\bar{u}\ |\ \bar{x} \vdash\textit{Closed}\ t_2
                                                                       \end{array}} &
  \infer{\Delta;\bar{u}\ |\ \bar{x} \vdash\textit{Closed}\ \mathsf{Nil}}{\ }
  \\ \\
  \infer{\Delta;\bar{u}\ |\ \bar{x} \vdash\textit{Closed}\ \mathsf{Cons}\ t_1\ t_2}
         {\begin{array}{l}\Delta;\bar{u}\ |\ \bar{x} \vdash\textit{Closed}\ t_1\\
                          \Delta;\bar{u}\ |\ \bar{x} \vdash\textit{Closed}\ t_2
         \end{array}} &
  \infer{\Delta;\bar{u}\ |\ \bar{x} \vdash \textit{Closed}\ \mathsf{R}\ t_1\ t_2\ t}
        {\begin{array}{l}
         \Delta;\bar{u}\ |\ \cdot \vdash\textit{Closed}\ t_1\\
         \Delta;\bar{u}\ |\ \bar{x} \vdash\textit{Closed}\ t_2\\
         \Delta;\bar{u}\ |\ \bar{x} \vdash\textit{Closed}\ t
         \end{array}}
   &\infer{\Delta;\bar{u}\ |\ \bar{x}\vdash\textit{Closed}\ \mathbf{f}[\bar{t}]}
        {\begin{array}{l}
             (\mathbf{f}[\bar{u}] = t) \in \Delta \\
            |\bar{u}| = |\bar{t}| \\
           \Delta;\bar{u}\ |\ \cdot \vdash \textit{Closed}\ \bar{t}
         \end{array}} 
\\ \\
\infer{\Delta;\bar{u}\ |\ \bar{x}\vdash \textit{Closed}\ \cdot}{\ }&
\infer{\Delta;\bar{u}\ |\ \bar{x}\vdash \textit{Closed}\ (t,\bar{t})}
  {\begin{array}{l}
    \Delta;\bar{u}\ |\ \bar{x}\vdash \textit{Closed}\ t \\
    \Delta;\bar{u}\ |\ \bar{x}\vdash \textit{Closed}\ \bar{t}
\end{array}}&\ 
   \end{array}
  \]
  \caption{Well-formedness of definitions and lists of definitions; closed schematic terms and lists of such terms}
\label{fig:wf}
\end{figure}

\subsection{Example}

We may define higher-order functions like \texttt{map} using a schematic definition:
\[
\texttt{map}[f] = \lam{x}{\mathsf{R}\ x\ (\lam{h}{\lam{t}{\lam{r}{\mathsf{Cons}\ (f\ h)\ r}}})\ \mathsf{Nil}}
\]
\noindent Call this definition $D_{\texttt{map}}$.  We have $\textit{Wf}\ D_{\texttt{map}}$, because the
body of the definition, namely
\[
\mathsf{R}\ x\ (\lam{h}{\lam{t}{\lam{r}{\mathsf{Cons}\ (f\ h)\ r}}})\ \mathsf{Nil}
\]
\noindent is closed with respect the parameter $f$.  Being closed
means that the term iterated by \textsf{R} should have no free
variables, but may use the parameter $f$, as indeed it does.

\subsection{Reduction}





\section{Typing without termination}

\bibliographystyle{plain} \bibliography{main}

\end{document}

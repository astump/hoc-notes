\documentclass{article}

\usepackage{amsmath,amssymb,amsthm}
\usepackage{url}
\usepackage{hyperref}
\usepackage{proof}
\usepackage{stmaryrd}
%\usepackage{MnSymbol}
\usepackage{parskip}
\usepackage{fullpage}
\usepackage{mathpartir}
\usepackage{subcaption}
\usepackage{float}


\begin{document}

\newcommand{\interp}[1]{\llbracket #1 \rrbracket}
\newcommand{\rinterp}[1]{\llparenthesis #1 \rrparenthesis}
\newcommand{\lft}[1]{\llceil #1 \rrceil}
\newcommand{\gprj}[1]{| #1 |}
\newtheorem{theorem}{Theorem}
\newtheorem{lemma}[theorem]{Lemma}
\newtheorem{corollary}[theorem]{Corollary}
\newtheorem{definition}[theorem]{Definition}
\newtheorem{notation}[theorem]{Notation}
\newtheorem{proposition}[theorem]{Proposition}
%\theoremstyle{definition}
\newtheorem{example}{Example}[section]
\newtheorem{remark}{Remark}[section]
\newcommand{\lftcc}[0]{$\lft{\texttt{CC}}$\xspace}
\newcommand{\infcc}[0]{$\texttt{CC}_\infty$\xspace}
\newcommand{\Pia}[3]{\Pi\, #1: #2\, .\, #3}
\newcommand{\Pik}[4]{\Pi\, #1\langle #2 \rangle : #3\, .\, #4}
\newcommand{\all}[2]{\forall\, #1\,.\, #2}
\newcommand{\alll}[3]{\forall^{#1}\, #2\,.\, #3}
\newcommand{\lam}[2]{\lambda\, #1\, .\, #2}
\newcommand{\tlam}[3]{\lambda\, #1 : #2\, .\, #3}
\newcommand{\Tlam}[3]{\Lambda\, #1 : #2\, .\, #3}
\newcommand{\capl}[0]{\stackrel{\cdot}{\cap}}
\newcommand{\ltl}[0]{\lambda 2^{\mathcal{L}}}
\newcommand{\tth}[3]{\theta\, #1 : #2\, .\, #3}
\newcommand{\univ}[0]{\mathcal{U}}
\newcommand{\subst}[3]{[#1 / #2] #3}
\newcommand{\betaeq}[2]{#1 \mathop{=_{\beta}} #2}
\newcommand{\reftc}[1]{\RefTirName{#1}}
\newcommand{\wrec}[2]{#1\ \bullet\ #2}
\newcommand{\ctor}[3]{[ #1 , #2, #3 ]}
\newcommand{\wunit}[0]{\langle\rangle}

\newcommand{\sar}[0]{SAR\xspace}
\newcommand{\eqtm}[3]{#2\{#1 \mathrel{\operatorname{=}} #3\}}
\newcommand{\rfl}[1]{\epsilon\ #1}
\newcommand{\cng}[4]{\rho\, [#1\,.\,#2]\ #3 - #4}
\newcommand{\ext}[1]{\xi\, #1}

\newcommand{\rrec}[0]{\mathsf{R}_{\textit{Nat}}}
\newcommand{\w}[3]{\langle #1 , #2 , #3 \rangle}
\newcommand{\wb}[3]{\ll #1 , #2 , #3 \gg} 
\newcommand{\wc}[3]{\langle| #1 , #2 , #3 |\rangle}

\newcommand{\inj}[2]{\iota_#1\ #2}

\title{Proposal: a Type Theory for Bend}

\author{Aaron Stump}

\maketitle

\section{What is a type theory?}

A type theory is a statically typed programming language that can be
understood as a logic.  Programs are viewed as proofs, and the types
of programs are viewed as the formulas they prove.  This famous idea
is called the Curry-Howard isomorphism.  A very simple example is the
program $\lam{x}{\lam{y}{x}}$, which takes in input $x$ and then input
$y$, and returns $x$.  This program can be given the type $A \to B \to
A$, for any types $A$ and $B$.  That type expresses that the program
takes in an input of type $A$ and then one of type $B$, and returns a
result of type $A$.  That indeed correctly describes the behavior of
$\lam{x}{\lam{y}{x}}$.  But it is also a valid logical formula, where
the $\to$ operator is implication: $A$ implies $B$ implies $A$ for the
trivial reason that $A$ implies $A$, and adding an extra assumption
that $B$ is true does not change that fact.  To go beyond just
propositional logic, type theories use more expressive types than just
implications.  We will see examples below.

To be interpreted as a logic, it is not enough to have a way
to view the types of a programming language as formulas.  We must
ensure that it is not possible to prove false formulas.  So the
language must be logically sound.  In type theory, an essential part
of ensuring logical soundness is to guarantee that all programs
terminate.  The reason for this is that an infinite loop can be
viewed, in most programming languages, as having any type one wants.
So you can prove the formula \textsf{False} by writing a diverging
program.

Much theoretical effort has been expended on techniques for proving
logical soundness of type theories, by showing that all programs
are guaranteed to terminate.  Bend has a very interesting original
approach to this problem, which we consider next.

\section{Bend's approach to logical soundness}

There are two current traditions for devising type theories, that
should be mentioned for comparison with Bend's approach:

\begin{enumerate}
\item \textbf{Church-style} type theory builds up a notion of typed
  terms (programs), where the types are inherent to those terms.  By a
  difficult argument, one shows that all well-typed programs
  terminate.  So the type system is enforcing termination, in addition
  to other properties usually enforced by static typing.  From
  termination, it is then easy to argue that the system is logically
  sound.  This is because it is relatively easy to show that values,
  which are the final results of computation, cannot have type False.

\item \textbf{Curry-style} type theory starts with a notion of
  type-free program, and then adds types to describe properties of the
  behavior of programs.  For example, the identity function can be
  described as having type $X \to X$ for any type $X$, as it is
  guaranteed to take an input of type $X$ and return an output of type
  $X$ (namely, the input it was given).  A difficult argument is still
  required to show that typing enforces termination.  But the language
  design is made quite a bit easier by not having types be inherent
  parts of programs.  This is because in reasoning about programs,
  one does not then have to reason about types inside them.  Programs
  are type-free, and typing comes second.  In fact, the slogan I propose
  for this style of type theory is ``Computation First'' (because we
  first explain what type-free programs are and how they execute, and
  only afterwards use types to describe their properties).  The great
  computer scientist Jean-Louis Krivine puts it simply: ``types can
  be thought of as properties of $\lambda$-terms''~\cite[page 43]{krivine93}.
\end{enumerate}

Bend's philosophy can be viewed as a strengthened form of Curry-style
type theory, with the modified slogan: ``Terminating Computation
First''.  The idea, proposed by Victor Taelin, is similar to
Curry-style type theory, where one first defines type-free programs,
and how they compute.  But differently, these programs are designed so
that they are guaranteed to terminate, without reference to any notion
of typing.  Just the structure of the programs and the rules for how
the execute are sufficient to establish that all programs terminate.
Giving a detailed proof of that fact is still not trivial, but
expected to be much simpler than the approaches based on typing.  And
then one has a lot of freedom to design a type system on top of the
terminating type-free language.  Now the only requirement is that the
language should have the usual type-safety property that ones expects
of any statically typed programming language. This is vastly easier to
achieve than crafting a type system that enforces termination.

As we begin our look at Bend, it is important to emphasize one further
design goal for the language: minimality.  We aim to have a core
language with a small number of primitive operations, and similarly
for the type system.  This is both for ease of implementation, and
reliability of the language design.

\section{Bend's core language}

The syntax of Bend's core programming language is shown in
Figure~\ref{fig:pl}.  Programs, called terms, can be variables $x$,
machine integers $i$, or labels $l$.  Labels will be used to
distinguish between different kinds of data, and they could be
implemented just by machine integers as well.  There are also infix
applications $t\ o\ t'$ of arithmetic operators $o$ to arguments $t$
and $t'$.

Terms can also be anonymous
functions $\lam{x}{s}$, with the restriction that $x$ may be used at
most once in $s$.  Some restriction is needed, or else it is very easy
to write diverging $\lambda$-terms.  Traditionally, type theories have
restricted anonymous functions using types.  With Bend's approach to
logical consistency, we need a type-free way to enforce termination of
$\lambda$-terms.  One method is to restrict how often a variable may
be used in an anonymous function.  Bend requires $\lambda$-bound
variables to be used at most once.  Such $\lambda$-abstractions are
called affine.  It is well known that this restriction ensures
termination.  It does impose serious limits on programs written with
anonymous functions, but we will see that the way recursion works
expands the possibilities greatly (while preserving termination).

Returning to the syntax: we have applications $t\ t'$ of a term $t$
being used as a function to term $t'$ given as the argument to that
function.  We have a trivial piece of data $\wunit$, which is useful as
a placehold.  We could use a machine integer $i$ as a base case
instead, but we will see that with typing, it is more convenient to
have a separate trivial piece of data.  That is $\wunit$.  We have a
way to form structured data $\ctor{l}{n}{r}$, and a term $\wrec{r}{t}$
for recursing over such data.  These constructs constitute a version of
what is known as W-types, and they will be presented in detail below.
Finally, there is a label-matching function $\{ l_1 \mapsto t_1\ ;\ \ldots\ ;\ l_k\mapsto t_k \}$,
which will return term $t_i$ if applied to label $l_i$. 


\begin{figure}
  \[
  \begin{array}{llll}
    \textit{Variables}  & x,y,z,\ldots & \ &\ \\
    \textit{Labels}  & l & \ &\ \\ 
    \textit{Machine integers}  & i,j,\ldots & ::= & 0\ |\ 1\ | \ \cdots \\
    \textit{Arithmetic operators} & o & ::= & +\ |\ *\ |\ \cdots \\        
    \textit{Terms} & s,r,t & ::= & x\ |\ i\ |\ l\ |\ t\ o\ t'\ |\ \lam{x}{s}\ |\ t\ t'\ |\ \wunit\ |\ \ctor{l}{n}{r}\ |\ \wrec{r}{t} \\
    \ &\ &\ & \{ l_1 \mapsto t_1\ ;\ \ldots\ ;\ l_k\mapsto t_k \}
  \end{array}
\]
\caption{The syntax for Bend's programming language.  In $\lambda$-abstractions, the variable $x$ is allowed
  to occur at most once in the body $s$}
\label{fig:pl}
\end{figure}

\subsection{Structured data}

To implement data structures, every programming language needs some
approach to creating and processing structured data.  In Bend, the
main form of structured data is the construction $\ctor{l}{n}{r}$.
Before we discuss this, though, it is very helpful to have a way of
constructing pairs $(t,t')$.

\subsubsection{Pairs}

There are several possibilities for how to include pairs in Bend.
Here we propose to $\lambda$-encode them.  So we take this definition:
\[
(t,t') := \lam{c}{c\ t\ t'}
\]
\noindent Syntactically, we propose to add pairs as primitive syntax,
but the implementation of Bend will just treat them as defined.  There
are two benefits to defining pairs this way.  First, we do not need to
add another primitive construction for pairs to the semantics of the
language: they are just defined using a $\lambda$-term.  Second, we
gain affine access to the components of a pair: we can make use of a
pair just once and still obtain both its components.  If instead we
had primitive accessors like $p.1$ and $p.2$ for accessing the
components of a pair $p$, we would not be able to write simple
functions like the one that swaps the components of a pair, as an
affine function.  For such a function would be written as
\[
\lam{p}{(p.2,p.1)}
\]
\noindent where we can see that the $\lambda$-bound input variable is used twice.
Instead, we define swapping of pair $p$ as
\[
p\ \lam{x}{\lam{y}{(y,x)}}
\]
\noindent This looks a little peculiar, but recall that pairs are defined to be functions.
So if $p$ is $(1,2)$, for example, then we will have this computational behavior:
\[
(1,2)\ \lam{x}{\lam{y}{(y,x)}} \ =\ (\lam{c}{c\ 1\ 2})\ \lam{x}{\lam{y}{(y,x)}}\ \leadsto\ (\lam{x}{\lam{y}{(y,x)}})\ 1\ 2 \leadsto^* (2,1)
\]
\noindent The input variable $c$ gets instantiated with
$\lam{x}{\lam{y}{(y,x)}}$, giving that $\lambda$-term access to both
components of the pair.  These $\lambda$-abstractions are all linear:
the input variables $c$, $x$, and $y$ are used exactly once in the
terms where they are $\lambda$-bound.

\subsubsection{W-structures}
\label{sec:wstruct}

The construction $\ctor{l}{n}{f}$ will be called a W-structure (or just structure, for short).
It is the basic form in Bend for recursively structured data.  The idea for structures
comes from the type-theoretic construction known as W-types.  Indeed, Bend's structures
are just a version of those for W-types.  Since Bend starts from a type-free programming
language and then adds types, we start with W-structures, and define the W-types that
describe them later.  We also have a primitive feature $\wrec{r}{t}$ of the language,
for recursively processing a structure.

A structure $\ctor{l}{n}{f}$ consists of three parts:
\begin{itemize}
\item a \emph{selector} $s$, which indicates what kind of structure this is
\item nonrecursive subdata $n$, which will not be recursively processed
  by recursions $\wrec{r}{t}$
\item a recursive subdata function $f$, which takes in an index $x$
  that specifies which piece of recursive subdata is desired, and
  returns it.
  \end{itemize}

\noindent The index $x$ given to the subdata function $f$ is drawn
from some finite set if the structure has only finitely many immediate
subdata.  For example, a list node has one piece of subdata, namely
the tail, while a node of a binary tree has two pieces of subdata.  So
the indices would come from a one-element set and a two-element set,
respectively.  But the power of W-structures comes from the fact that
$i$ could also be from an infinite set, in which case the structure
can have infinitely many immediate subdata.  There is still a
restriction, though.  While there may be infinitely many paths through
a structure, each path is finite.  This enables us to define
$\wrec{r}{t}$ as a form of terminating recursion over structures,
because we cannot recurse infinitely deeply into a structure.  Note
that for representing particular datatypes below, the selectors will
always just be labels, but the definition does not require this.

For recursions $\wrec{r}{t}$ over a structure $t$, we write a function
$r$ which takes the structure's selector, nonrecursive subdata, and
subdata function.  $r$ also is provided a function $q$ that returns
recursive results.  Then $r$  returns a result, generally by calling
$q$.  For each possible input to $f$, the function $q$ returns the
result of recursing on the subdata $f\ i$; that is, the $i$'th piece
of subdata.  The recursor $\wrec{r}{t}$ uses this function $r$ to
recursively process a structure $t$. The reduction rule, which
explains how recursions work computationally, is:

\[
\wrec{r}{\ctor{l}{n}{f}} \leadsto r\ l\ n\ f\ (\lam{x}{\wrec{r}{(f\ x)}})
\]

\noindent So someone using the recursor writes $r$, and then for each
piece of data $\ctor{s}{n}{f}$, that function $r$ will be invoked with
the selector $s$, the nonrecursive subdata $n$, and the function $f$,
which $r$ can then call as needed to obtain subdata.  The fourth
argument to $r$ is the value that $r$ will use for $q$, namely
$\lam{x}{\wrec{r}{(f\ x)}}$.  That function takes in an index $x$, and
recursively invokes the recursor on the $x$'th piece of subdata, given
by $f\ x$.  


\subsection{Reduction semantics}

Figure~\ref{fig:opsem} gives the complete reduction semantics for
Bend.  This semantics specifies the meaning of programs by saying how
they compute.  Calling it a ``reduction'' semantics indicates that we
are not specifying a deterministic evaluation order for programs.
There might be multiple choices of which part of a term to reduce
next, and the rules does not specify which one should be chosen.  So
the reduction relation is nondeterministic.  But we will show below
(Section~\ref{sec:opmeta}) that all choices are guaranteed to lead to
the same result.  It could happen that some choices lead to that
result more efficiently.  But they all will, in principle, succeed.

Returning to Figure~\ref{fig:opsem}: there are reduction rules for
reducing terms where an anonymous function $\lam{x}{s}$ is applied to
an argument $t$.  This is the well-known $\beta$-rule from lambda
calculus.  It uses the notation $[t/x]s$ to denote the result of
substituting $t$ for $x$ in $s$.  This should be done in a standard
way to avoid capturing free variables of $t$ as one passes under
$\lambda$-abstractions in $s$.  There is also the reduction rule for
W-structures, mentioned above.  The semantics as presented here does
not specify the exact behavior of the machine operations on integers.
Those can be specified in detail later as needed.  Finally, there is a
reduction rule for applying a label-matching function $\{ l_1 \mapsto
t_1\ ;\ \ldots\ ;\ l_k\mapsto t_k \}$ to a label $l_i$, where
$i\in\{1,\ldots,k\}$.  The term reduces to $t_k$ in that case.
Applying a label-matching function to any other argument does not
reduce.  It is a stuck term, and we will impose typing
(Section~\ref{sec:tp} below) to prevent such terms from arising.

The two rule inference rules at the bottom of the figure then define
multi-step reduction $\leadsto^*$ in terms of single-step reduction
$\leadsto$.  The first rule says that if you can single-step reduce
$s$ to $t$, then you can do a multi-step reduction of a term
containing $s$ some finite number of times to one containing instead
$t$.  The term $r$ mentioned in the rule might not contain $x$ at all,
in which case we get reflexivity of $\leadsto^*$, as the rule will say
that $r \leadsto^* r$ in that situation.  It is noteworthy that the
semantics overall is quite compact, in terms of numbers of rules
needed to define it.  This is in keeping with Bend's goal of
minimality.

\begin{figure}
  \[
\begin{array}{l}
  \begin{array}{lll}
    (\lam{x}{s})\ t & \leadsto & [t/x]s \\
    \wrec{r}{\ctor{s}{n}{f}} & \leadsto & R\ s\ n\ f\ (\lam{x}{\wrec{r}{(f\ x)}}) \\
    i\ o\ i' & \leadsto & j\ \ \textnormal{according to machine semantics for }o \\
    \{ l_1 \mapsto t_1\ ;\ \ldots\ ;\ l_k\mapsto t_k \}\ l_i & \leadsto & t_i\ \ \textnormal{if } i\in[1,k]
  \end{array}
  \\ \\
  \begin{array}{lll}
  \infer{[s/x]r \leadsto^* [t/x]r}{s \leadsto t} &\ \ &
  \infer{r \leadsto^* t}{r \leadsto^* s & s \leadsto^* t}
  \end{array}
\end{array}  
\]
  \caption{Reduction semantics for Bend}
  \label{fig:opsem}
  \end{figure}

\section{Typing}
\label{sec:tp}

Having proposed a syntax and reduction semantics for Bend, let us now
consider typing.  We can describe Bend's proposed type system very
succinctly: it is the (Curry-style) Calculus of Constructions plus a
simple variant of W-types, and including also a primitive typed
extensional equality type.  In this section, we will elaborate on
this short description, one ingredient at a time.

One overarching point: the type system is Curry-style, meaning that
typing rules should be viewed as defining a typing relation on the
untyped terms whose syntax is given above in Figure~\ref{fig:pl}.  But
in this section, we will present the rules using annotated terms,
which contain enough information to confirm a typing for a term.  An
example is the annotated term $\tlam{x}{A}{t}$, with a type for the
bound variable.  So the typing rules look like Church-style rules,
computing types for terms that have various type annotations in them.
But there is a critical difference.  With Curry-style typing, we are
entitled to make use of an erasure function $|e|$ to drop all
annotations from the term parts of expressions $e$.  This function is
used in the conversion rule, allowing us to change the type $A$ of a
term to a type $B$, if $|A|$ and $|B|$ are convertible.  Using erasure
at conversion allows us to equate many more terms than with
Church-style typing, and eliminates many difficult technical problems
reasoning about annotated terms, which are encountered in Church-style
type theory.


\subsection{The Calculus of Constructions}

\begin{figure}
\[
\begin{array}{lllll}
  \textit{TyVar} & \ni & X,Y,Z & \ &\ \\
  \textit{TmVar} & \ni & x,y,z & \ &\ \\\\
  \textit{Knd} & \ni & \kappa & ::= & \star\ |\ \Pia{x}{T}{\kappa}\ |\ \Pia{x}{\kappa}{\kappa'} \\
  \textit{Ty} & \ni & R,S,T & ::= &
     X \ | \ \Pia{x}{T}{T'}\ |\ \Pia{X}{\kappa}{T}\ |\ \tlam{x}{T}{T'} \ |\ \tlam{X}{\kappa}{T}\ |\ T\ t\ |\ T\ T' \\\\
  \textit{Tm} & \ni & r,s,t & ::= & x \ |\ \tlam{x}{T}{t}\ | \ \tlam{X}{\kappa}{t}\ |\ t\ t'\ |\ t\ T \ 
\end{array}
\]
\caption{Syntax of CC}
\label{fig:ccsyn}
\end{figure}

The syntax for the Curry-style Calculus of Constructions (CC) is
listed in Figure~\ref{fig:ccsyn}.  The three main syntactic categories
defined there are annotated terms (\textit{Tm}), types (\textit{Ty},
and kinds (\textit{Knd}).  Types abstract terms, and kinds abstract
types.  For example, every natural number is a term, abstracted by the
type \textit{Nat}.  And every type (like \textit{Nat}) is abstracted
by kind $\star$.  We also use meta-variable $e$ for any form of expression,
whether term, type, or kind.  The typing rules are given in Figure~\ref{fig:tc}.

Erasure, used in the conversion rules \textsc{conv} and
\textsc{t-conv}, is defined in Figure~\ref{fig:erase}.  Erasure leaves
the structure of types and kinds unchanged, but drops typing
annotations from terms.  The conversion rules also reference a
$\beta$-equality relation on kinds and types, respectively.  This
relation is defined in Figure~\ref{fig:convcc}, which also defines
$\beta$-reduction for the two possible kinds of type-level
$\beta$-redexes of CC (one where the argument is a term, the other
where it is a type).

These typing rules for CC do not enforce that $\lambda$-abstractions
are affine.  This requirement is to be imposed on the erasures on
terms, and it is checked syntactically, without reference to typing.
This is so that the termination of the system does not depend on
typing.

\begin{figure}
  \begin{subfigure}[c]{1.0\textwidth}
    \begin{mathpar}
      \inferrule[type]{ }
                      {\Gamma \vdash \star}

      \inferrule[tk-pi]{\Gamma \vdash T : \star \\ \Gamma, x : T \vdash \kappa}
                       {\Gamma \vdash \Pia{x}{T}{\kappa}}

      \inferrule[kk-pi]{\Gamma \vdash \kappa \\ \Gamma, X : \kappa \vdash \kappa'}
                       {\Gamma \vdash \Pia{X}{\kappa}{\kappa'}}
    \end{mathpar}
    \caption{Classifiable kinds}
  \end{subfigure}
  %%%%%%%%%%%%%%%%%%%%
  \begin{subfigure}[c]{1.0\textwidth}
    \begin{mathpar}
      \inferrule[t-var]{X : \kappa \in \Gamma}
                       {\Gamma \vdash X : \kappa}

      \inferrule[t-conv]{\Gamma \vdash T : \kappa \\ \betaeq{|\kappa|}{|\kappa'|}}
                        {\Gamma \vdash T : \kappa'}
      \\
      \inferrule[tt-pi]{\Gamma \vdash T : \star \\ \Gamma, x : T \vdash T' : \star}
                       {\Gamma \vdash \Pia{x}{T}{T'} : \star}

      \inferrule[kt-pi]{\Gamma \vdash \kappa \\ \Gamma, X : \kappa \vdash T : \star}
                       {\Gamma \vdash \Pia{X}{\kappa}{T} : \star}
      \\
      \inferrule[tk-abs]{\Gamma \vdash \Pia{x}{T}{\kappa} \\ \Gamma, x : T \vdash T' : \kappa}
                        {\Gamma \vdash \tlam{x}{T}{T'} : \Pia{x}{T}{\kappa}}

      \inferrule[kk-abs]{\Gamma \vdash \Pia{X}{\kappa}{\kappa'} \\ \Gamma, X : \kappa \vdash T : \kappa'}
                        {\Gamma \vdash \tlam{X}{\kappa}{T} : \Pia{X}{\kappa}{\kappa'}}
      \\
      \inferrule[tk-app]{\Gamma \vdash T : \Pia{x}{R}{\kappa} \\ \Gamma \vdash t : R}
                        {\Gamma \vdash T\ t : \subst{t}{x}{\kappa}}

      \inferrule[kk-app]{\Gamma \vdash T : \Pia{X}{\kappa}{\kappa'} \\ \Gamma \vdash T' : \kappa}
                        {\Gamma \vdash T\ T' : \subst{T'}{X}{\kappa'}}
    \end{mathpar}
    \caption{Kinding}
  \end{subfigure}
  %%%%%%%%%%%%%%%%%%%%
  \begin{subfigure}[c]{1.0\textwidth}
    \begin{mathpar}
      \inferrule[var]{x : T \in \Gamma}
                     {\Gamma \vdash x : T}

      \inferrule[conv]{\Gamma \vdash t : T \\ \betaeq{|T|}{|T'|}}
                      {\Gamma \vdash t : T'}
      \\
      \inferrule[tt-abs]{\Gamma \vdash \Pia{x}{T}{T'} : \star \\ \Gamma, x : T \vdash t : T'}
                        {\Gamma \vdash \tlam{x}{T}{t} : \Pia{x}{T}{T'}}

      \inferrule[kt-abs]{\Gamma \vdash \Pia{X}{\kappa}{T} : \star \\ \Gamma, X : \kappa \vdash t : T}
                       {\Gamma \vdash \tlam{X}{\kappa}{t} : \Pia{X}{\kappa}{T}}
      \\
      \inferrule[app]{\Gamma \vdash t : \Pia{x}{T}{T'} \\ \Gamma \vdash t' : T}
                        {\Gamma \vdash t\ t' : \subst{t'}{x}{T'}}

      \inferrule[t-app]{\Gamma \vdash t : \Pia{x}{\kappa}{T'} \\ \Gamma \vdash T : \kappa}
                        {\Gamma \vdash t\ T : \subst{T}{x}{T'}}
    \end{mathpar}
    \caption{Typing}
    \label{fig:tc-atm}
  \end{subfigure}
  %%%%%%%%%%%%%%%%%%%%
  \caption{Typing Derivations for CC}
  \label{fig:tc}
\end{figure}

\begin{figure}
  \[
  \begin{array}{lll}
    |\star| & = & \star \\
    |\Pia{x}{T}{\kappa}| & = & \Pia{x}{|T|}{|\kappa|} \\
    |\Pia{x}{\kappa}{\kappa'}| & = & \Pia{x}{|\kappa|}{|\kappa'|} \\
    |X| & = & X \\
    |\Pia{x}{T}{T'}| & = & \Pia{x}{|T|}{|T'} \\
    |\tlam{x}{T}{T'}| & = & \tlam{x}{|T|}{|T'|} \\
    |T\ t| & = & |T|\ |t| \\
    |T\ T'| & = & |T|\ |T'| \\
    |t \simeq t'| & = & |t| \simeq |t'| \\
    |x| & = & x \\
    |\tlam{x}{T}{t}| & = & \lam{x}{|t|} \\
    |\tlam{X}{\kappa}{t}| & = & |t| \\
    |t\ t'| & = & |t|\ |t'| \\
    |t\ T| & = & |t| 
  \end{array}
  \]
  \caption{Erasing annotations from terms, types, and kinds of CC}
  \label{fig:erase}
\end{figure}

\begin{figure}
\begin{mathpar}
    \inferrule[conv-ttbeta]{\ }{(\tlam{x}{T}{T'})\ t \leadsto^* [t/x]T'}
    
    \inferrule[conv-ktbeta]{\ }{(\tlam{X}{\kappa}{T'})\ T \leadsto^* [T/X]T'}

\\ \\

    \inferrule[conv-red]{e \leadsto^* e'}{e =_\beta e'}

    \inferrule[conv-sym]{e =_\beta e'}{e' =_\beta e}

    \inferrule[conv-subst]{e_1 =_\beta e_2}{[e_1/x]e =_\beta [e_2/x]e}

\end{mathpar}
  \caption{Reduction for type-level $\beta$-redexes, and conversion}
  \label{fig:convcc}
\end{figure}

\subsection{Enumerated types}

We extend the syntax of CC defined in Figure~\ref{fig:ccsyn} above, with the following additions:
\[
\begin{array}{lllll}
  \textit{Labels} & \ni & l &\ &\ \\
  \textit{Ty} & \ni & R,S,T & ::= & \ldots\ |\ \{ l_1, \ldots, l_k \}\ |\ \{ l_1 \mapsto T_1\ ;\ \ldots\ ;\ l_k \mapsto T_k \} \\
  \textit{Tm} & \ni & r,s,t & ::= & \ldots\ |\ l\ |\ \{ l_1 \mapsto t_1\ ;\ \ldots\ ;\ l_k \mapsto t_k \}
\end{array}
\]
\noindent Then type-level reduction is extended:
\[
\{ l_1 \mapsto T_1\ ;\ \ldots\ ;\ l_k \mapsto T_k \} \ l_i \ \leadsto\ T_i\ \ \textnormal{if } i \in [1,k]
\]
\noindent Typing and kinding are extended in Figure~\ref{fig:enumtp}.
Label-matching functions at the term level are dependently typed: in
the \textsc{match} rule, the return type of the function is allowed to
depend on the input label $x$.  We do not have a similar dependency
for label-matching functions at the type level, because CC does not
have kind-level $\lambda$-abstractions.

\begin{figure}
  \begin{mathpar}
    \inferrule[enum]{\ }{\Gamma \vdash \{l_1,\ldots,l_k \} : \star}

    \inferrule[t-match]{\all{i\in[1,k]}{\Gamma\vdash T_i : \kappa}}
                       { \{ l_1 \mapsto T_1\ ;\ \ldots\ ;\ l_k \mapsto T_k \} :  \{ l_1, \ldots, l_k \} \to \kappa}
  
    \inferrule[match]{\all{i\in[1,k]}{\Gamma\vdash t_i : T\ l_i}}
              { \{ l_1 \mapsto t_1\ ;\ \ldots\ ;\ l_k \mapsto t_k \} :  \Pia{x}{\{ l_1, \ldots, l_k \}}{T\ x}}
  \end{mathpar}
\caption{Typing and kinding for enumerated types}
\label{fig:enumtp}
\end{figure}



\subsection{W-types}

To give types to the W-structures described in
Section~\ref{sec:wstruct} above, Bend uses a form of W-types,
introduced by Martin-L\"of~\cite{Mar84} (see also the exposition
in~\cite{mlttprog}).  Our syntax is further extended as so:
\[
\begin{array}{lllll}
  \textit{Ty} & \ni & R,S,T & ::= & \cdots\ |\ \w{L}{N}{R} 
\end{array}
\]
\noindent This new primitive type form is meant to abstract
W-structures $\ctor{s}{n}{f}$, where (to recall) $s$ is a selector
indicating which constructor is used, $n$ is the nonrecursive data,
and $f$ is a function returning subdata.  To type such a construction,
we clearly need:
\begin{itemize}
\item a type $S$ for the selector $s$
\item a type constructor $N$ where $N\ s$ is the type for the nonrecursive data
\item a type constructor $R$ where $R\ s$ is the type for inputs to the function $f$
\end{itemize}
\noindent $N$ and $R$ are type constructors (that is, functions returning types), because
we need to describe the types for each possible selector $s$.  So given $s$, we need
to know what the type will be for nonrecursive data.
In the case of the recursive subdata, only the input type of $f$ needs to be specified, because the output
type is determined already: given an index (like the natural number $i$ selecting the $i$'th smaller
ordinal for a \textit{Limit}), $f$ returns a piece of subdata, which recursively
has the W-type again.  The proposed syntax is $\w{S}{N}{R}$.

Following the above discussion, $\w{S}{N}{R}$ should have kind $\star$
(so it is a type) assuming
\begin{itemize}
\item $S$ has kind $\star$
\item $N$ and $R$ both have kind $S \to \star$
\end{itemize}

\noindent If one wanted to ensure that the top-level syntactic form of an expression
determined whether that expression is a term, type, or kind, then it would be necessary
to pick a slightly different syntax, maybe $\wb{S}{N}{R}$ or (if one avoids Unicode) $\wc{S}{N}{R}$.
In my opinion, it is rather nicer just to reuse the term syntax, if possible, the way
Haskell does for tuples as values and tuple types (for example, $(\textit{True},\textit{"hi"}) : (\textit{Bool},\textit{String})$).

The typing rule for W-types as just described would then be:
\[
\infer{ \Gamma \vdash \w{S}{N}{R} : \star}{\Gamma \vdash S : \star & \Gamma \vdash N : S \to \star  & \Gamma \vdash R : S \to \star }
\]
\noindent This records exactly the same information as in the informal
description: the three premises just say what we wrote above, that $S$
has kind $\star$ ($S : \star$), and so forth.  

\subsubsection{Typing rule for W-structures}

Now let us see the typing rule for constructions $\ctor{s}{n}{f}$.  For this
to have a W-type of the form $\w{S}{N}{R}$ we were just discussing, what has
to be true?

\begin{itemize}
\item The selector $s$ certainly has to have the type $S$.
\item The nonrecursive data $n$ should have the type $N\ s$ that
  the type constructor says should be there if the selector is $s$.
\item The function $f$ needs to take in a value of type $R\ s$,
  because this is the type for indices into the collection of subdata,
  and such an index is what $f$ expects as input. It should return a
  result of the W-type, namely the subdata for the given index.
\end{itemize}

Formalizing these ideas, we get the following typing rule:
\[
\infer{\Gamma \vdash \ctor{s}{n}{f} : \w{S}{N}{R}}
      {\Gamma \vdash s : S & \Gamma \vdash n : N\ s & \Gamma \vdash f : R\ s \to \w{S}{N}{R}}
      \]
      
\subsubsection{Typing rule for recursion}

A recursion $\wrec{r}{t}$ takes in a function $R$ and a value $t$ of the W-type.
Recalling the computation rule will help us understand what the typing of a recursion
should be:
\[
\wrec{r}{\ctor{s}{n}{f}} \ \ = \ \ R\ s\ n\ f\ (\lam{x}{\wrec{r}{(f\ x)}})
\]
\noindent We know from the typing rule for constructions that for some $S$, $N$, and $R$, we will have
\[
\begin{array}{lll}
  s & : & S \\
  n & : & N\ s \\
  f & : & R\ s \to \w{S}{N}{R}
\end{array}
\]
\noindent Since $R$ takes in those inputs on the right-hand side of the computation rule,
we know that the type of $R$ must look like:
\[
\Pia{s}{S}{\Pia{n}{N\ s}{\Pia{f}{R\ s \to \w{S}{N}{R}}{\cdots}}}
\]
\noindent We just need to fill in the missing part of this type.  We expect that
the recursion is going to compute a value of some type constructor $C$ which might
be a function of the input value $\ctor{s}{n}{f}$.  So $C$ will have kind
\[
\w{S}{N}{R} \to \star
\]
\noindent This is similar to how a recursion on natural numbers $n$ can compute a value of type $P\ n$,
which is used, for example, when reasoning by induction on $n$.  Now what type will the final (fourth)
input of $R$ on the right-hand side of the computation rule have?  That input is
\[
\lam{x}{\wrec{r}{(f\ x)}}
\]
\noindent It is taking in a value of type $R\ s$, just as the subdata function $f$ does.  And it returns
a value of type $C$ (since it is recursing).  $C$ depends on the input to the recursion, so in this case,
since $f\ x$ is the input, the result of the recursion will have type $C\ (f\ x)$.  This means that the
type for $\lam{x}{\wrec{r}{(f\ x)}}$ is
\[
\Pia{x}{R\ s}{C\ (f\ x)}
\]
\noindent Putting all this together, the type of $R$ should be:
\[
\begin{array}{l}
  \Pia{s}{S}{\ }\\
  \Pia{n}{N\ s}{\ }\\
  \Pia{f}{R\ s \to \w{S}{N}{R}}{\ } \\
  (\Pia{x}{R\ s}{C\ (f\ x)}) \to \\
  C\ \ctor{s}{n}{f}
  \end{array}
\]
\noindent Finally, the type of $\wrec{r}{d}$ is $C\ d$.  So if we think of $\wrec{r}{d}$
as a proof by induction, it is saying that to prove a property $C$ of some value $d$ of
type $\w{S}{N}{R}$, it is sufficent to assume

\begin{itemize}
\item $s : S$
\item $n : N\ s$
\item $f:R\ s \to \w{S}{N}{R}$
\end{itemize}

\noindent and show $C\ \ctor{s}{n}{f}$, assuming for induction hypothesis
\[
\Pia{x}{R\ s}{C\ (f\ x)}
\]
\noindent That type expresses that $C$ holds for all the subdata that $f$ can produce.

Putting this all into one rather indigestible rule, we have:
\[
\infer{\Gamma \vdash \wrec{r}{d} : C\ d}
      {
        \begin{array}{lll}
\Gamma \vdash S : & \star \\
\Gamma \vdash N : & S \to \star \\
\Gamma \vdash R : & R \to \star \\          
\Gamma \vdash r : & \Pia{s}{S}{\ }  \\
\ &  \Pia{n}{N\ s}{\ }  \\
\ &  \Pia{f}{R\ s \to \w{S}{N}{R}}{\ }  \\
\ &  (\Pia{x}{R\ s}{C\ (f\ x)}) \to  \\
\ &  C\ \ctor{s}{n}{f} \\
\Gamma\vdash d : & \w{S}{R}{R}
        \end{array} }
      \]



\subsection{An extensional equality type}

We continue our extension of the syntax of types and type-annotated terms with the following additions:
\[
\begin{array}{lllll}
  \textit{Ty} & \ni & R,S,T & ::= & \cdots\ |\ \eqtm{t}{T}{t'} \\
  \textit{Tm} & \ni & r,s,t & ::= & \cdots\ |\ \rfl{t}\ |\ \cng{x}{T}{t}{t'}\ |\ \ext{t}
\end{array}
\]
\noindent The type $\eqtm{t}{T}{t'}$ expresses extensional equality of
terms $t$ and $t'$ at type $T$.  Syntactically, it binds more tightly
than $\Pi$, so an expression like $\Pia{x}{T}{\eqtm{s}{T'}{t}}$ should
be parsed as $\Pia{x}{T}{(\eqtm{s}{T'}{t})}$.  The new term constructs
are for inference rules about this equality: reflexivity ($\rfl{t}$),
congruence ($\cng{x}{T}{t}{t'}$), and extensionality ($\ext{t}$).  

The typing and kinding rules are shown in
Figure~\ref{fig:eqtp}.  The first rule states that $\rfl{t}$ proves
that $t$ equals itself at its type.  The second rule, for typing
terms of the form $\cng{x}{T}{s}{t}$, allows us
to change the type of some term $t$ by replacing chosen occurrences
of $t_1$ with $t_2$, when we have a proof $s$ that $t_1$ and $t_2$
are equal at type $T'$.  The occurrences to be replaced are indicated
using a variable $x$ to show where in some type $T$ the occurrences
of $t_1$ are to be exchanged for $t_2$.  That variable $x$ has to
have the same type $T'$ for typing $T$, otherwise $T$ might be using
$t_1$ and $t_2$ at different types than the ones for which they are
proved equal.  For example, if we prove that $t_1$ equals $t_2$ at type
$\textit{False} \to T$, where $\textit{False}$ is uninhabited, then
we are not allowed to replace $t_1$ except at that same type.  Otherwise
we could unsoundly change $t_1$ to $t_2$ that is not equivalent at the
type where it is used.

The last three rules are for typing uses $\ext{t}$ of extensionality.
Two of them express the idea that if functions are equal for all
inputs, then they are equal at function type.  There is one rule for
equality of term inputs, and another for equality of type inputs.
Since we work in a Curry-style theory, the premise of the rule for
equality of type inputs just requires that $t_1$ equals $t_2$, because
in Curry-style type theory, we do not apply terms to type arguments.
Church-style type theory would have had $\Gamma \vdash t :
\Pia{X}{\kappa}{\eqtm{t_1\ X}{T}{t_2\ X}}$ instead.  The very last
rule of the figure expresses extensionality for equality types
themselves, by stating that all proofs of equality are themselves
equal: $t_1$ and $t_2$ are both proofs of some equation
$\eqtm{r}{T}{s}$, then they are equal at that type.  It should be
noted that this principle is incompatible with Homotopy Type Theory,
which leaves open the possibility of other proofs of equalities
besides just reflexivity~\cite{hottbook}.  If desired, the axiom could
be omitted.

Finally, we extend the erasure function to the new term constructs
for equality, as follows:
\[
\begin{array}{lll}
  |\rfl{t}| & = & \lam{x}{x} \\
  |\cng{x}{T}{s}{t}| & = & |t| \\
  |\ext{t}| & = & |t|
\end{array}
\]
\noindent The critical point here is that when equality proofs are
used to change the type of a term, with the $\cng{x}{T}{s}{t}$
construct, the proof $s$ of the equality is discarded by erasure.
This makes a huge difference in working with terms where equality
proofs are used to change the types of subterms, because in
Church-style type theory, one can end up with stuck coercions: an
equality proof is being used to change the type of some subterm, but
it ends up blocking reduction or reasoning about the term.  This can
happen if the equality proof either is an assumption (i.e., a
variable) or derived from one.  One could have a stuck coercion around
a $\lambda$-abstraction, for example, that is being applied to an
argument.  The coercion would block the $\beta$-reduction step,
because it is in between the function and the argument.  With the
proposed approach, however, the equality proofs are all erased, and
such stuck terms do not arise.


\begin{figure}
  \[
  \begin{array}{lll}
    \infer{\Gamma\vdash\eqtm{s}{T}{t} : \star}{\Gamma\vdash s : T & \Gamma\vdash t : T}
    &\ &
\infer{\Gamma\vdash\rfl{t} : \eqtm{t}{T}{t}}{\Gamma\vdash t : T}
\\\\    
\infer{\Gamma\vdash\cng{x}{T}{s}{t} : [t_2/x]T}
          {\Gamma\vdash s : \eqtm{t_1}{T'}{t_2} & \Gamma \vdash t : [t_1/x]T & \Gamma,x:T'\vdash T : \star}
    &\ &
    \infer{\Gamma\vdash\ext{t} : \eqtm{t_1}{(\Pia{x}{T'}{T})}{t_2}}{\Gamma \vdash t : \Pia{x}{T'}{\eqtm{t_1\ x}{T}{t_2\ x}}}
\\ \\
\infer{\Gamma\vdash\ext{t} : \eqtm{t_1}{(\Pia{X}{\kappa}{T})}{t_2}}{\Gamma \vdash t : \Pia{X}{\kappa}{\eqtm{t_1}{T}{t_2}}}
&\ &
\infer{\Gamma\vdash\ext{t} : \eqtm{t_1}{(\eqtm{r}{T}{s})}{t_2}}{\Gamma\vdash t_1 : \eqtm{r}{T}{s} & \Gamma\vdash t_2 : \eqtm{r}{T}{s}}
    \end{array}
  \]
\caption{Typing and kinding rules for equality}
\label{fig:eqtp}
  \end{figure}
  
\section{Examples}

%% for the example above of \verb|Lim| for ordinals, the
%% name would be a natural number $i$, and the subdata would be the
%% $i$'th ordinal contained in that limit ordinal.  Let us write
%% $\ctor{l}{f}$ for this construction with label $l$ and function $f$.
%% So it is effectively a pair, but the second argument is always a
%% function.


\subsection{Symmetry and transitivity of equality}



\section{Metatheory of Bend's programming language}
\label{sec:opmeta}

\section{Metatheory of Bend's type system}
\label{sec:ttmeta}

\bibliographystyle{plain}
\bibliography{main}

\end{document}
